\section{Architecture} \label{sec:arch}

The architecture of the bladeRF consists primarily of a LimeMicro LMS6002D RF
transceiver (RF section), an Altera Cyclone IV FPGA (Baseband section), and a
Cypress FX3 USB 3.0 controller (Transport section).

It's recommended to read this section with a copy of the
\href{https://nuand.com/bladerf.pdf}{schematic}\cite{BLADERF_SCHEMATIC}.

\subsection{RF Section} \label{sec:arch-rf}
The RF connections to the bladeRF are two SMA connections RX (J53) and TX (J54).
Each of the RF sections are separated out into a low-band covering 300~MHz
through 1.5~GHz and a high-band covering 1.5~GHz through 3.8~GHz.  The low-band
input goes into RXLNA1 of the LMS6002D whereas the high-band goes into RXLNA2.
The low-band output comes from TXOUT1 and the high-band output comes from
TXOUT2.  Switching between the two bands is accomplished using an AS211-334
SPDT which is controlled by the FPGA.

For reception, internal to the LMS6002D is a zero-IF homodyne receiver
architecture with over 70~dB of variable receive gain, tunable anti-aliasing
filters with bandwidths from 1.5~MHz to 28~MHz, and pair of 12-bit ADCs for
digitizing the baseband signal.  For external observation, header J71 connects
to the analog differential output pins of the LMS6002D.  Pins 1 and 6 connect
to IP and IN, respectively.  Pins 3 and 4 connect to QP and QN, respectively.

\subsection{Baseband Section} \label{sec:arch-bb}
The FPGA is responsible for receiving and transmitting digital baseband samples
as well as command and control of the LMS6002D, Si5338, VCTCXO DAC and RF
switching.

The sample interface on the LMS6002D side is a simple 12-bit interface which
transfers the IQ signal at twice the sample rate.

The command and control interface for the LMS6002D is a 20~MHz SPI connection
which is controlled using a NIOS soft CPU inside the FPGA.

\subsection{Transport Section} \label{sec:arch-transport}
The FX3 is utilized as a transport bridge between the USB connection and the
FPGA.  Samples are communicated over the 32-bit GPIF-II interface whereas
command/control is communicated over the UART.

\subsection{Clocking} \label{sec:arch-clocking}
The bladeRF uses a 1.5~ppm VCTCXO that has a fundamental frequency of 38.4~MHz.
This clock is sent through a 1:2 buffer feeding the FX3 oscillator input as
well as a Silicon Labs Si5338 clock generator.  The Si5338 distributes a
free-running 38.4~MHz clock to the LMS6002D to be used as a frequency reference
and to the Cyclone IV to be used as a system clock.

The Si5338 is also responsible for creating both the TX and RX sampling clocks,
both of which are independent from each other.  The RX clock is fed to the
LMS6002D and fed back to the FPGA to produce a source-synchronous clocked
interface.  The TX clock is considered system-synchronous since the
FPGA is fed a clock with the same phase as the LMS6002D.

The bladeRF is also capable of supplying a clock to the SMB connector at J62.
The purpose of this clock is for distributing a common 38.4~MHz between
multiple devices, or for generating a 10~MHz reference clock to lock external
test equipment to the bladeRF.  The bladeRF cannot accept a 10~MHz input on
J62.

\subsection{Frequency Accuracy} \label{sec:arch-accuracy}
The bladeRF has an on-board 16-bit DAC for fine tuning (trimming) the on-board
VCTCXO.  During factory calibration, the SMB output is fed a buffered 38.4~MHz
clock from the VCTCXO and an external frequency counter is used to measure the
actual frequency.  An algorithm is used to find the DAC value which yields a
measured frequency of 38.4~MHz $\pm 1$~Hz yielding a calibration within 26~ppb.

Due to crystal aging and temperature variation versus calibration, the value
of the DAC may have to change for frequency sensitive setups.

