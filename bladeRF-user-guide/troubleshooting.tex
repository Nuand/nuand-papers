\section{Troubleshooting} \label{sec:troubleshooting}

This section provides troubleshooting advice and solutions to common issues.

\subsection{Getting Help}\label{sec:help}

For issues not in this document, one may seek help through a number of other
avenues:

{\footnotesize
\begin{itemize}
    \item The \textit{Troubleshooting} forum \cite{TROUBLESHOOTING}
    \item IRC: \#bladeRF on Freenode \cite {FREENODE}
    \item Contacting Nuand
        \begin{itemize}
            \item \textit{Via email:} bladeRF@Nuand.com
            \item \textit{Via the Web:} \url{https://www.nuand.com/contact.php}
        \end{itemize}
\end{itemize}
}

\subsubsection{Information to Provide}
Below are some important of pieces of information to provide when asking
questions.

{\footnotesize
\begin{itemize}
    \item Information about the host machine
    \begin{itemize}
        \item Are you using a VM? What vitalization software and version?
        \item Are you using an embedded system, smartphone, or tablet? What specific device are you using?
        \item Are you using a USB 2.0 or USB 3.0 port? At which speed did the device connect?
        \item What USB host controller are you using?
    \end{itemize}

    \item Information about the host OS
    \begin{itemize}
        \item What version are you using? (\textit{\textbf{Linux users}: What distribution is it?})
        \item Is the OS 32-bit or 64-bit?
    \end{itemize}

     \item Software
     \begin{itemize}
         \item What \libbladerf, FPGA bitstream, and \fx3 firmware versions are in use?
         \item Where was the software obtained (\textit{e.g., from source, from a package manager})
         \item If using GNU Radio and/or gr-osmosdr, what are their corresponding versions?
     \end{itemize}

    \item Procedural issues
    \begin{itemize}
        \item What is the source of the procedure you are following? (\textit{e.g., website, guide, blog post, forum post})
        \item What steps have you already performed? Which one is causing issues?
    \end{itemize}

    \item Logs and Output
    \begin{itemize}
        \item Note any warnings or error messages.
        \item Gather verbose output from \bladerfcli using the \cmd{-v verobse} command-line option.
        \item Gather verbose output from \libbladerf-based programs by setting the \var{BLADERF\_LOG\_LEVEL} environment variable to \var{verbose}.
        \item \texttt{Linux Users:} Make a note of any relevant information in \cmd{dmesg} output.
    \end{itemize}
\end{itemize}
}

\subsection{Configuration and Build Issues}

\subsubsection{Warnings treated as errors}

By default, the host software is build with \texttt{-Werror} or \texttt{/WX},
with the intent of forcing developers and contributors to address warnings.

If you find that host software is not building due to a warning being treated
as an error, please contact the \bladerf developers to report this. Be sure
to accompany verbose build logs and your compiler version.

As a temporary workaround, you may configure the build to disable this behavior:
\begin{lstlisting}[style=snippet]
    cmake -DTREAT_WARNINGS_AS_ERRORS=OFF <path to CMakeLists.txt>
\end{lstlisting}

\subsubsection{Missing libusb}

\textit{\textbf{Linux and \osx}:}
If \cmake reports that libus is not found, it may be the case that
\program{pkg-config} could not find the library and/or its development headers.
Ensure the libusb development package has been installed and that the library
is in the standard library path or a location specified by the
\var{PKG\_CONFIG\_PATH} environment variable.

\textit{\textbf{\windows}:}
Verify that the \var{LIBUSB\_PATH} variable provided to \cmake is correctly assigned to
the location where libusb has been downloaded and extracted.

\subsubsection{Outdated libusb}

If the following message is printed when running \cmake, a newer libusb version
must be installed.

\begin{lstlisting}[style=snippet]
Failed to compile (compiled=0) or run (retval=255) libusb version check.
This may occur if libusb is earlier than v1.0.10.
Setting LIBUSB_VERSION to 0.0.0
\end{lstlisting}

libusb v1.0.10 or later is required to build. The latest available version is
\textbf{highly} recommended, due to numerous fixes and improvements (or added
support) for USB 3.0 and various OSes. See the libusb \fname{ChangeLog} file
\cite{LIBUSB_CHANGELOG} for more information.

\subsection{Runtime Issues}

\subsubsection{Unable to Find/Open Device: Linux}\label{sec:open-linux}

If unable to open the \bladerf in Linux, it may be the case that the current
user does not have sufficient permissions to access the device.

As of \libbladerf v1.5.0, a warning will be printed if this is the case and
associated API functions will return a status of \var{BLADERF\_ERR\_PERMISSION}.

\begin{lstlisting}[style=numbered-snippet, caption=Insufficient permissions to probe device]
$ bladeRF-cli -p

[WARNING @ libusb.c:292] Found a bladeRF via VID/PID, but could not open
                         it due to insufficient permissions.

  probe: No devices are available. If one is attached, ensure it
         is not in use by another program and that the current
         user has permission to access it.

\end{lstlisting}

\begin{lstlisting}[style=numbered-snippet, caption=Insufficient permissions to open device]
[WARNING @ libusb.c:428] Found a bladeRF via VID/PID, but could not open
                         it due to insufficient permissions.

No bladerf device(s) available.

If one is attached, ensure it is not in use by another program
and that the current user has permission to access it.
\end{lstlisting}

If using an older version, note that using running
\cmd{bladeRF-cli -v verbose} will print out more information about error codes,
and that setting \var{BLADERF\_LOG\_LEVEL=debug} in your environment will
provide additional debug output.

In order to access the \bladerf, Linux users must ensure:
%
\begin{itemize}
    \item The \fname{88-nuand.rules} file has been installed to \fname{/etc/udev/rules.d/}
    \item When building from source, the CMake option \var{INSTALL\_UDEV\_RULES} defaults to \var{ON}.
    \item The user is in the group specified by the CMake variable \var{BLADERF\_GROUP}.
\end{itemize}


The \var{BLADERF\_GROUP} defaults to \var{plugdev}, because Debian/Ubuntu users are
typically in this group when allowed to access removable USB storage devices.
You may wish to change this to conform to your security policies or
distribution.  The udev rules have been reloaded since the installation of the
aforementioned rules file. This can be done via executing the following
command and then re-plugging the device:

\centerline{\cmd{sudo udevadm control --reload-rules}}

If you have verified the above items and still cannot see the device:

\begin{itemize}
    \item Ensure the device is powered, as indicated by a lit LED \texttt{D1}.
    \item Confirm that you see the device in the output of \cmd{lsusb -v}
    \item Look for any information about the device's connection in the output
          \program{dmesg}
    \item Look for any failure messages in the log from the command:\\
          \cmd{bladeRF-cli -e version -v verbose}
    \item Examine and test the above for both USB 2.0 and USB 3.0 ports, if possible.
\end{itemize}

\subsubsection{Unable to Find/Open Device: \windows}\label{sec:open-windows}

To find and open the \bladerf in \windows, either a \libusb-based
driver or \cypress \cyusb driver must be installed. These drivers
may be installed using the \bladerf \windows installer
\cite{BLADERF_WINDOWS_INSTALL_GUIDE}. Alternatively, one may use \zadig
\cite{ZADIG} to install a \libusb-based driver.

Once a driver has been installed, ensure that the device is listed as
being associated with this driver in Device Manager. If this is the case
and the device still cannot be opened:

\begin{itemize}
    \item Ensure the device is powered, as indicated by a lit LED \texttt{D1}.
    \item Look for any failure messages in the log from the command:\\
          \cmd{\bladerf-cli -e version -v verbose}
    \item Examine and test the above for both USB 2.0 and USB 3.0 ports, if possible.
\end{itemize}

\subsubsection{Unable to Find/Open Device: \osx}\label{sec:open-osx}

If a device cannot be opened in \osx, first verify that the device is
powered on (which may be observed via LED \texttt{D1}) and is
is detected by the system:

\begin{lstlisting}[style=snippet, caption=\cmd{system\_profiler} output showing connected \bladerf]
$ system_profiler SPUSBDataType

USB:

    USB 3.0 Bus:

      Host Controller Driver: AppleUSBXHCIPPT
      PCI Device ID: 0x1e31
      PCI Revision ID: 0x0004
      PCI Vendor ID: 0x8086

        bladeRF:

          Product ID: 0x6066
          Vendor ID: 0x1d50
          Version: 0.00
          Serial Number: 9d1698a25946a1ce2876ab5953b45fb
          Speed: Up to 5 Gb/sec
          Manufacturer: Nuand
          Location ID: 0x14600000 / 10
          Current Available (mA): 1800
          Current Required (mA): 200
          Extra Operating Current (mA): 300

    USB 3.0 Bus:

      Host Controller Driver: AppleUSBXHCIPCI
      PCI Device ID: 0x1142
      PCI Revision ID: 0x0000
      PCI Vendor ID: 0x1b21
\end{lstlisting}

If the device appears in \cmd{system\_profiler} output:

\begin{itemize}
    \item Look for any failure messages in the log from the command:\\
          \cmd{bladeRF-cli -e version -v verbose}
      \item Look for any connection-related information in the output of: \cmd{sudo dmesg}
    \item Examine and test the above for both USB 2.0 and USB 3.0 ports, if possible.
    \item Reinstall \libusb from \macports or \brew. It has been reported that after
        performing a system upgrade, conflicts between the \osx \libusb and a version
        installed via \macports or \brew result in the inability to open a \bladerf.
        Reinstalling \libusb from the associated third-party package manager has
        been found to be a solution to this.
\end{itemize}

\subsubsection{Permissions Errors with PyBOMBS-based Install}\label{sec:open-linux}
\pybombs \cite{PYBOMBS} allow users to perform installations as non-root users.
To support this, it builds \bladerf support using the \cmake argument,
\var{-DINSTALL\_UDEV\_RULES=OFF}.  This prevents the installation from
attempting to copy \program{udev} rules to \fname{/etc/udev/rules.d/}.  As a
result, users may see the permissions errors
shown in Section \ref{sec:open-linux}.

Therefore, \pybombs users may need to (have their administrator) install the
\fname{88-nuand.rules} \cite{BLADERF_UDEV} file:

\begin{lstlisting}[style=numbered-snippet, caption=Manually installing udev rules]
# Download this file from the source repository
$ wget https://github.com/Nuand/bladeRF/blob/master/host/misc/udev/88-nuand.rules.in
$ mv 88-nuand-rules.in 88-nuand.rules

# Edit the file to replace @BLADERF_GROUP@ with a group your user is in.
$ vi 88-nuand.rules

# Change its permissions and install it
$ sudo chmod 644 88-nuand.rules
$ sudo mv 88-nuand.rules /etc/udev/rules.d/

# Reload the rules. Unplug and re-attach the device after running this.
$ sudo udevadm control --reload-rules
\end{lstlisting}

\subsubsection{Failure to Use Device on Embedded Platforms}

Some embedded platforms may not be able to supply a sufficient amount of
power to the \bladerf via the USB connection alone (especially on USB 2.0). The
symptoms of the device being underpowered vary. The device may
constantly enumerate in bootloader mode, or the device fails to open while
attempting to load its FPGA bitstream.

Therefore, when working with embedded platforms, it is recommended to either:
\begin{itemize}
    \item Use a USB 3.0 port, if available.
    \item Use an \textbf{externally powered} USB 3.0 hub.
    \item Power the \bladerf through its DC barrel jack instead of the USB connection.
\end{itemize}

\subsubsection{Multiple Programs Attempting to Access the Device}

\libbladerf allows only one process to access the device at any given time. If
an application unexpectedly reports that no \bladerf devices are available, ensure
that no applications are currently holding open device handles.

When applications are holding a handle to a \bladerf, \texttt{LED2} on the
\bladerf will blink. This provides a quick visual means of checking whether
another process was left using the device.

\subsubsection{Failures on USB 3.0 only}

If a \bladerf operates correctly on a USB 2.0 connection, but not on a USB 3.0
connection, it may be the case that: \begin{itemize}
    \item \textbf{\libusb}: A newer \libusb version may be required. See the
        \libusb CHANGELOG \cite{LIBUSB_CHANGELOG} for notes regarding USB 3.0
        support on your operating system.

    \item \textbf{Linux}: Driver or kernel updates may be required. For example,
        a 3.12 patch \cite{LINUX_SGLIMIT_PATCH} was reported to resolve
        FPGA-loading issues on USB 3.0 with certain host controllers.. Other
        fixes applied after the 3.13 kernel release reportedly fixed issues
        such as poor transfer rates and system-wide lockups when using a
        Renesase uPD720202-based
        controller.

    \item The USB 3.0 controller being used may require a firmware or driver
        update.  Early XHCI controllers are known to have a number of defects.
        Check the chipset or card manufacturer's website for updates.

        Should USB 3.0 issues be found to be associated to a particular chipset, it is
        recommended to upgrade to at \textit{least} an Intel 7 Series XHCI. (Later
        series have been found to yield improved perfomance.)

\end{itemize}

\subsubsection{libbladerf Messages About Outdated Firmware or FPGA}

\libbladerf will print warnings if it detects that either firmware executing on the \fx3
or the bitstream loaded in the FPGA are too old to allow for proper operation.

Additionally, \libbladerf will return the \var{BLADERF\_ERR\_SUPPORT} status
code when API functionality is not available with the firmware or FPGA version being
used. Enabling the debug log-level will often provide additional information in
this case.

See Section \ref{sec:fw-upgrade} for more information about upgrading the \fx3 firmware,
and the Nuand FPGA download page \cite{BLADERF_FPGA} for the latest FPGA version.

\subsubsection{Messages About Outdated libbladeRF}

\libbladerf prints one the following messages if the \fx3 firmware or FPGA bitstream
versions in use supersedes the version of libbladeRF that is in use:

\begin{lstlisting}[style=snippet]
    FPGA version (vX.Y.Z) is newer than entries in libbladeRF's compatibility table.
    Please update libbladeRF if problems arise.
\end{lstlisting}

\begin{lstlisting}[style=snippet]
    Firmware version (vX.Y.Z) is newer than entries in libbladeRF's compatibility table.
    Please update libbladeRF if problems arise.
\end{lstlisting}

This message is simply intended to let the user know that the firmware or FPGA
versions being used have been developed and tested with a newer version of \libbladerf.
In most cases, \libbladerf will operate successfully with newer versions of
these items. However, it is generally recommended to upgrade \libbladerf to
obtain fixes and new features.

\subsubsection{bladeRF-cli Reports Device is in Bootloader Mode}

When starting up, \bladerfcli detects devices in this bootloader mode
and prints a message alerting users to this: \\

\begin{lstlisting}[style=snippet]
    NOTE: One or more FX3-based devices operating in bootloader mode
          were detected. Run 'help recover' to view information about
          downloading firmware to the device(s).
\end{lstlisting}

See Section \ref{sec:recovery} for more information about recovering from
this state.

\subsubsection{Devices Times Out While Loading FPGA}

Attempting to load the wrong (or corrupted) FPGA image to the bladeRF results
in \libbladerf returning the \var{BLADERF\_ERR\_TIMEOUT} status code. Verify
that you are using the correct image for your device.

If you are unsure as to which size FPGA is on your \bladerf, identify the
Altera Cyclone IV package on the top of the \bladerf. The chip package should
read \textit{EP4CExF23C8N}, where \textit{x} is either 40 or 115. Use the
\fname{hostedx40.rbf} file for the former, and \fname{hostedx115.rbf} for the latter.

\subsubsection{Failure to find libbladeRF.so, libbladeRF.dylib, or bladeRF.dll}

If a program fails to run, printing a message that \fname{libbladeRF.so} (Linux),
\fname{libbladeRF.dylib} (OSX), or \fname{bladeRF.dll} (\windows) cannot be found,
it may be the case that the system does not have \libbladerf in its library search path.

\textbf{Linux and OSX}:
If you have build \bladerf software from source, but are unsure where it has
been installed, refer to the \fname{install\_manifest.txt} file within your
build directory, and/or the \var{CMAKE\_INSTALL\_PREFIX} variable in
\fname{CMakeCache.txt}

\textbf{Linux}: \\
Ensure that \libbladerf is installed in a location specified in one of the files
in \fname{/etc/ld.so.conf.d/} or create a new file in this directory specifying the
location of \libbladerf. Check the output of \program{ldd <your program>} to
determine where the program is attempting to find \libbladerf. Using
\var{LD\_LIBRARY\_PATH} may be helpful in debugging library path issues, but
is not recommended as a permanent solution.


\textbf{\osx}: \\
Generally, applications in OSX utilize \texttt{@rpath} to establish paths to
shared libraries. Use \cmd{otool -L <your program>} to check the location where
the program is attempting to find \libbladerf. Using \var{DYLIB\_LIBARARY\_PATH}
may be helpful in debugging library path issues, but is not recommended as a
permanent solution.

\textbf{\windows}: \\
Ensure that the folder containing \program{bladeRF.dll} is listed in the \var{PATH}
environment variable. If you have updated \var{PATH} (either manually or via the
\bladerf installer), you may need to log out and log back in before the change
takes effect.
