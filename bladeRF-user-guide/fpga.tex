\section{FPGA} \label{sec:fpga}

The FPGA is a very flexible baseband processing element to the bladeRF.
Modifying the FPGA is highly encouraged and welcomed.

\subsection{Building the FPGA Image} \label{sec:fpga-building}
Describe how to build the FPGA image here.

\subsection{Modification} \label{sec:fpga-mods}
The bladeRF Quartus II project is separated into different revisions.  The
preferred method to building a modification is to create and build a new
revision of the FPGA you wanted to modify.

\subsubsection{Adding A New Revision} \label{sec:fgpa-newrev}
Steps to creating a new revision in the project.
Steps for creating the QIP file.
Steps for modification of the source.

\subsection{Clock Domains} \label{sec:fpga-clocks}
The FPGA has four main clock domains: 80~MHz NIOS II clock, 100~MHz GPIF clock,
RX clock, TX clock.  The RX and TX clocks, while they may be the same
frequency, are considered asynchronous to each other due to their unknown
phase relationship.

\subsection{NIOS II Soft CPU} \label{sec:fpga-nios}
The NIOS II Soft CPU inside the FPGA mainly handles command and control of the
bladeRF.  Commands and responses are sent via a UART operating at 4~Mbaud in
an 8-N-1 configuration.

Specifics of the command packet can be found in the software section.

\subsection{GPIF Interface} \label{sec:fpga-gpif}
The GPIF interface is a 32-bit bidirectional interface running at 100~MHz.
This interfaces with the \texttt{fx3\_gpif} block in the FPGA.

For transfers flowing from the FPGA to the GPIF, the FPGA first checks whether
there is enough data for a full DMA transfer.  The DMA transfer size is
determined by the firmware running in the FX3.  By default, each transfer over
the GPIF is 512 32-bit words (2048 bytes) when the USB speed is SuperSpeed, and
256 32-bit words (1024 bytes) when the USB speed is High.  The bladeRF cannot
operate at any other speed.

\begin{center}
    \begin{tikztimingtable}
        [timing/d/background/.style={fill=white}, timing/lslope=0.2]
            PCLK    & L 10{T}         ;[dotted] 4L; 10{T} \\
            RX REQ  & L H 7H H H     ;[dotted] 4H; 5H 5L   \\
            RX ACK  & L L H 6H H H   ;[dotted] 4H; 5H 5L   \\
            DATA    & U U D 6D D D   ;[dotted] 4D; 5D 5U   \\
        \extracode
            \begin{pgfonlayer}{background}
                \begin{scope}[semitransparent,semithick]
                    % \vertlines[gray]{2.1,4.1,...,21.1} % Falling edge
                    \vertlines[gray]{1.1,3.1,...,9.1, 15.1, 17.1,...,23.1} % Rising edge with skip over dot
                \end{scope}
            \end{pgfonlayer}
    \end{tikztimingtable}
\end{center}

Example timing diagram.

